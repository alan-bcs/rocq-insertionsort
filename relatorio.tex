\documentclass{article}

\usepackage[utf8]{inputenc}
\usepackage[portuguese]{babel}
\usepackage[T1]{fontenc}
\usepackage{lmodern}
\usepackage{fullpage}
\usepackage[usenames]{color}
\usepackage[color]{coqdoc}
\usepackage{url}
\usepackage{makeidx,hyperref}
\usepackage[all]{xy}
\usepackage{soul,xcolor}

\usepackage{mathpartir}
\usepackage{tcolorbox}
\usepackage{amsmath,amssymb}

\newcommand{\tto}{\twoheadrightarrow}
\newcommand{\ott}{\twoheadleftarrow}

\newcommand{\flavio}[1]{{\color{red}#1}}
\newcommand{\cflavio}[2]{\st{#1}{\color{red} #2}}
\newcommand{\rflavio}[1]{\st{#1}}
\newcommand{\mtulio}[1]{{\color{blue}#1}}

\title{A Correção do Algoritmo de Ordenação por Inserção}
\author{Alan dos Santos Dias - 232007830 \\ Bruno Henrique Duarte - 221022239 \\ João Marcos Rodrigo Cardoso - 232027411}
\date{\today}

\begin{document}
\maketitle
\setstcolor{red}

\input{insertion_sort.v}

\section{Conclusão}

A formalização apresentada demonstrou mecanicamente que o algoritmo 
Insertion Sort satisfaz as propriedades de ordenação e permutação. 
O uso da indução estrutural e a decomposição em casos (baseada na comparação entre elementos) 
permitiram cobrir exaustivamente todos os cenários de execução. A verificação pelo Coq elimina a 
possibilidade de erros lógicos comuns em provas manuais, garantindo que a lista de saída 
é, inequivocamente, uma versão ordenada da lista de entrada.


\begin{thebibliography}{9}

\bibitem{coqdoc}
THE COQ DEVELOPMENT TEAM.
\textit{The Coq Proof Assistant Reference Manual}.
INRIA. Disponível em: \url{https://rocq-prover.org/docs}. Acesso em: nov. 2025.

\bibitem{paanotes}
DE MOURA, Flávio L. C.
\textit{Notas de aula da disciplina Logica Computacional 1}.
Universidade de Brasília (UnB). Disponível em: \url{https://flaviomoura.info}. Acesso em: nov. 2025.

\end{thebibliography}

\end{document}
